		\begin{center}
			Министерство образования и науки РФ\\
			\textsc{ГОУ ВПО Рыбинский государственный авиационно-\\
			технический университет имени П.А. Соловьева}
		\end{center}
		\vspace{1em}
		\begin{center}
			Лаборатория БЖД\\
			Работа выполнена 19.02.2012\\
			Студентами гр. ИВП-09\\
			Виноградовой З.С., Кулаевским Д.Ю., Смирновым Н.Н.
		\end{center}
		\begin{center}
			\textsc{\bfseries отчет}\\
			\textsc{по лабораторной работе}\\
			"<Определение температуры вспышки горючих жидкостей">
		\end{center}

Цель работы: изучение методики определения температуры вспышки горючих жидкостей, оценка пожарной опасности промышленных предприятий, выбор соответствующего электрооборудования.

\section{Результаты эксперимента}
В ходе эксперимента было произведено 2 замера температуры:
\begin{enumerate}
	\item $117^{\circ}$,
	\item $118^{\circ}$.
\end{enumerate}
Также было измеряно атмосферное давление: 755 мм рт. ст.

\section{Основные формулы}
Температура вспышки: $T_B=0{,}736T_K$, где $T_K$~--- температура кипения в $K^{\circ}$.

Истиная температура в градусах Цельсия с учетом атмосферного давления: $t_{\text{И}}=t+\Delta{}t$, где $t$~--- средняя температура вспышки, а $\Delta{}t=0{,}345(P-760)$~--- поправка на атмосферное давление ($P$~--- барометрическое давление при испытании, мм рт. ст.).

\section{Обработка экспериментальных данных}
Среднее значение температуры: $t_{cp}=(117+118)/2=117{,}5^{\circ}$.

Расчетная температура: $t_B=0{,}736T_K=384{,}93K=111{,}93 {}^{\circ}C$, где $T_K=523K$~--- температура кипения дизельного топлива.

Поправка на атмосферное давление: $\Delta{}t=0{,}345(755-760)=-1{,}725\approx2^{\circ}$.

Истиная температура вспышки: $t_{\text{И}}=117{,}5-2=115{,}5 {}^{\circ}C$

%
\rotatebox{90}{
\begin{minipage}{0.3\linewidth}
	%\begin{table}[H]
	\begin{longtable}[!h]{|c|c|p{4cm}|}
		\hline
		\begin{turn}{-90}2\end{turn}&\begin{turn}{-90}1\end{turn}&  № повтора  \\
		\hline
		\begin{turn}{-90}$117$\end{turn}&\begin{turn}{-90}$188$\end{turn}& Температура вспышки по показаниям термометра  \\
		\hline
		\multicolumn{2}{|c|}{$111{,}93$} & Расчетная $t_B$  \\
		\hline
		\multicolumn{2}{|c|}{$117{,}5$} & Среднее значение $t_B$  \\
		\hline
		\multicolumn{2}{|c|}{$755$} & Барометрическое давление  \\
		\hline
		\multicolumn{2}{|c|}{$-2$} & Поправка на барометрическое давление  \\
		\hline
		\multicolumn{2}{|c|}{$115{,}5$} & Истиная $t_B$  \\
		\hline
		\multicolumn{2}{|c|}{ГЖ} & ЛВЖ или ГЖ  \\
		\hline
		\multicolumn{2}{|c|}{В, Г} & Категория производства  \\
		\hline
		\multicolumn{2}{|c|}{П-1} & Класс пожароопасных зон  \\
		\hline
		\multicolumn{2}{|c|}{П-1} & Вид электрооборудования  \\
		\hline
		\multicolumn{2}{|c|}{IP44} & Степень защиты оболочек  \\
		\hline
		\multicolumn{2}{|c|}{5'X} & Степень защиты светильника \\
		\hline
	\end{longtable}
	%\end{table}
	\end{minipage}
}
%\end{table}
\section{Вывод}
Исследуемая жидкость относится к классу горючих жидкостей (т. к. температура вспышки превышает $61^{\circ}$). Помещение в котором находится данная жидкость относится к категории помещений по взрывоопасности В и Г (пожароопасное), а зона в этом помещении относится к классу пожароопасных зон П-1.
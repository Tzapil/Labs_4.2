\newpage
\section{Постановка задачи}
Разобрать и реализовать алгоритмы перевода цвета в различные цветовые модели.

\section{Карткие теоретические сведения}
Цветовая модель~--- термин, обозначающий абстрактную модель описания представления цветов в виде кортежей чисел, обычно из трёх или четырёх значений, называемых цветовыми компонентами или цветовыми координатами. Вместе с методом интерпретации этих данных (например, определение условий воспроизведения и/или просмотра — то есть задание способа реализации), множество цветов цветовой модели определяет цветовое пространство.

Некоторые цветовые модели используются для цветовоспроизведения, например воспроизведения цвета на экранах телевизоров и компьютеров, или цветной печати на принтерах. Используя явление метамерии, устройства цветовоспроизведения не воспроизводят оригинальный спектр изображения, а лишь имитируют стимульную составляющую этого спектра, что в идеале позволяет получить картину неотличимую человеком от оригинальной сцены.

Наиболее распространенной моделью является \textbf{RGB}. Изображение в данной цветовой модели состоит из трёх каналов. При смешении основных цветов (основными цветами считаются красный, зелёный и синий) — например, синего (B) и красного (R), мы получаем пурпурный (M magenta), при смешении зеленого (G) и красного (R) — жёлтый (Y yellow), при смешении зеленого (G) и синего (B) — циановый (С cyan). При смешении всех трёх цветовых компонентов мы получаем белый цвет (W).

В телевизорах и мониторах применяются три электронных пушки (светодиода, светофильтра) для красного, зелёного и синего каналов.

Цветовая модель RGB имеет по многим тонам цвета более широкий цветовой охват (может представить более насыщенные цвета), чем типичный охват цветов CMYK, поэтому иногда изображения, замечательно выглядящие в RGB, значительно тускнеют и гаснут в CMYK.

\textbf{CMYK}~--- другая цветовая модель, использующаяся в полиграфии для стандартной триадной печати. Схема CMYK обладает сравнительно с RGB меньшим цветовым охватом. Цвет в CMYK зависит не только от спектральных характеристик красителей и от способа их нанесения, но и их количества, характеристик бумаги и других факторов. Фактически, цифры CMYK являются лишь набором аппаратных данных для фотонаборного автомата или CTP и не определяют цвет однозначно.

Особенность CMYK в том, что в модели используются четыре цвета, первые три в аббревиатуре названы по первой букве цвета, а в качестве четвёртого используется чёрный. Одна из версий утверждает, что K~--- сокращение от англ. \textit{blacK}. Согласно этой версии, при выводе полиграфических плёнок на них одной буквой указывался цвет, которому они принадлежат. Чёрный не стали обозначать B, чтобы не путать с B (англ. blue) из модели RGB, а стали обозначать K (по последней букве). Профессиональные цветокорректоры работают с десятью каналами RGBCMYKLab, используя доступные цветовые пространства. Поэтому при обозначении CMYK как CMYB фраза "<манипуляция с каналом B"> требовала бы уточнения "<манипуляция с каналом B из CMYB">, что было бы неудобно.

Также существует модель \textbf{HSV} (англ. Hue, Saturation, Value~--- тон, насыщенность, значение) или \textbf{HSB} (англ. Hue, Saturation, Brightness~--- оттенок, насыщенность, яркость)~--- цветовая модель, в которой координатами цвета являются:
\begin{itemize}
	\item Hue~--- цветовой тон, (например, красный, зелёный или сине-голубой). Варьируется в пределах $0{-}360^{\circ}$, однако иногда приводится к диапазону $0{-}100$ или $0{-}1$. В Windows весь цветовой спектр делится на 240 оттенков (что можно наблюдать в редакторе палитры MS Paint), то есть здесь "Hue" приводится к диапазону $0{-}240$ (оттенок 240 отсутствует, так как он дублировал бы 0).
	\item Saturation~--- насыщенность. Варьируется в пределах $0{-}100$ или $0{-}1$. Чем больше этот параметр, тем "<чище"> цвет, поэтому этот параметр иногда называют чистотой цвета. А чем ближе этот параметр к нулю, тем ближе цвет к нейтральному серому.
	\item Value (значение цвета) или Brightness~--- яркость. Также задаётся в пределах $0{-}100$ и $0{-}1$.
\end{itemize}

Модель была создана Элви Реем Смитом, одним из основателей Pixar, в 1978 году. Она является нелинейным преобразованием модели RGB.

Цвет, представленный в HSV, зависит от устройства, на которое он будет выведен, так как HSV~--- преобразование модели RGB, которая тоже зависит от устройства. Для получения кода цвета, не зависящего от устройства, используется модель Lab.

Модель HSV часто используется в программах компьютерной графики, так как удобна для человека. Ниже указаны способы «разворачивания» трёхмерного пространства HSV на двухмерный экран компьютера.

Модель \textbf{YCrCb} не является абсолютным цветовым пространством, скорее, это способ кодирования информации сигналов RGB. Для систем отображения используются сигналы основных цветов RGB. Эти сигналы не являются эффективными для хранения и передачи изображений, так как они имеют большую избыточность. Поэтому перевод в систему YCrCb позволяет передать информацию о яркости с полным разрешением, а для цветоразностных компонент произвести субдискретизацию, то есть выборку с уменьшением числа передаваемых элементов изображения, так как человеческий глаз менее чувствителен к перепадам цвета. Это повышает эффективность системы, позволяя уменьшить поток видеоданных. Значение, выраженное в YCbCr будет предсказуемо, если первично использовались сигналы основных цветов RGB.

Модель YCrCb используется при сжатии изображения в формате JPEG. Следует отметить, что стандарт JPEG (ISO/IEC 10918-1) никак не регламентирует выбор именно YCbCr, допуская и другие виды преобразования (например, с числом компонентов, отличным от трёх), и сжатие без преобразования (непосредственно в RGB), однако спецификация JFIF (JPEG File Interchange Format, предложенная в 1991 году специалистами компании C-Cube Microsystems, и ставшая в настоящее время стандартом де-факто) предполагает использование преобразования RGB$\rightarrow$YCbCr.

\newpage
\section{Исходный код основных процедур программы}
\subsection{Преобразование из RGB в HSV}
\begin{lstlisting}
void transformRGB(uchar &r, uchar &g, uchar &b)
{
    int   maxc = MAX3(r,g,b),
          minc = MIN3(r,g,b),
          delta = maxc - minc;
    double V = maxc,
           H = 0,
           S = 0,
           cr,cb,cg;

    if( V != 0 )
        S = (double)delta/V;

    if(S == 0)
        H = 0;
    else
    {
        cr = (double)(V-r)/delta;
        cb = (double)(V-b)/delta;
        cg = (double)(V-g)/delta;

        if( r == V )
            H = cb - cg;
        else if(g == V)
            H = 2 + cr - cb;
        else
            H = 4 + cg - cr;

        H = H * 60;
    }

    if(H < 0)
        H += 360;
    S = S * 255;
    H = H * 0.7083;

    r = round(H); g = round(S); b = V;
}
\end{lstlisting}

\newpage
\subsection{Преобразование из RGB в YCrCb}
\begin{lstlisting}
void transformRGB(uchar &r, uchar &g, uchar &b)
{
    double Y = 0,Cr = 0,Cb = 0;

    Y = (0.299*r)+(0.587*g)+(0.114*b);
    Cr = 128 + (0.5*r)-(0.418688*g)-(0.081312*b);
    Cb = 128 - (0.168736*r)-(0.331264*g)+(0.5*b);

    r = Y; g = Cr; b = Cb;
}
\end{lstlisting}

\subsection{Преобразование из RGB в CMYK}
\begin{lstlisting}
void transformRGB(uchar &r, uchar &g, uchar &b, uchar &k)
{
    uchar c = 0, m = 0, y = 0;

    c = 255 - r;
    m = 255 - g;
    y = 255 - b;
    k = MAX3(c,m,y);
    c = (c - k);
    m = (m - k);
    y = (y - k);

    r = c; g = m; b = y;
}
\end{lstlisting}

\section{Выводы}
В ходе лабораторной работы были изучены основные цветовые модели и их особенности. Были реализованы алгоритмы перевода цвета из модели RGB в модели CMYK, HSV и YCrCb.

В ходе изучения теоретических сведений было выяснено, что разнообразие различных моделий существует, как следствие разнообразия устройств вывода графической информации. Так, например, модель RGB наиболее удобна для отображения графической информации на экранах мониторов, но совершенно не подходит для печати на бумаге, в этом случае используется модель CMYK.
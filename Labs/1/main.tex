\newpage
\section{Цель работы}
Ознакомиться с методикой моделирования звеньев с дробно-рациональными передаточными функциями 1-го порядка с помощью пакета СТЕМ.

\section{Задание по лабораторной работе}
Звено описывается дифференциальным уравнением
$$
	y'+ay=bg
$$
Для заданных значений a и b найти теоретически выходной сигнал блока $y(t)$, если входной сигнал имеет вид:
\begin{enumerate}
	\item $g(t) = 1(t)$,
	\item $g(t) = \delta(t)$,
	\item $g(t) = \sin\omega{}t$.
\end{enumerate}
Построить таблицы и графики выходного сигнала $y(t)$ для всех случаев. Получить логарифмические характеристики блока. Вычисления можно выполнять с использованием пакета MATHCAD.

\section{Дифференциальное уравнение и передаточная функция заданого блока}
\begin{align*}
	y'&=0{,}5y+x\\
	W(p)&=\frac1{p+0{,}5}
\end{align*}

\section{Расчет выходных сигналов}
\subsection{Единичная ступень}
Изображение по Лапласу для данного входного сигнала равно $G(p)=\frac1p$.

Известно, что изображение по Лапласу выходного сигнала равно $Y(p)=W(p)G(p)$, значит
\begin{align*}
	Y(p)&=\frac1{p+0{,}5}*\frac1p=\frac1{p(p+0{,}5)}\\
	y(t)&=2-2e^{-0{,}5t}
\end{align*}
В таблице~\ref{step_t} указаны значения функции, график функции $h(t)$ изображен на рисунке~\ref{step}.

\begin{table}[h!]
	\caption{Таблица значений функции $h(t)$}
	\label{step_t}
	\scalefont{.7}
	\begin{tabular}{*{14}{|c}|}
	\hline 
	t&$0$&$0{,}5$&$1$&$1{,}5$&$2$&$2{,}5$&$3$&$3{,}5$&$4$&$4{,}5$&$5$&$5{,}5$&$6$ \\
	\hline
	$y(t)$&$0$&$0{,}4423$&$0{,}7869$&$1{,}0552$&$1{,}2642$&$1{,}4269$&$1{,}5537$&$1{,}6524$&$1{,}7293$&$1{,}7892$&$1{,}8358$&$1{,}8721$&$1{,}9004$ \\
	\hline
\end{tabular}		
\end{table}

\begin{figure}[h!]
		\centering
		\includegraphics[scale=.7]{step.jpg}
		\caption{График переходной характеристики $h(t)$}
		\label{step}
\end{figure}

\subsection{Импульс Дирака}
Изображение по Лапласу для импульса Дирака:$G(p)=1$.

Известно, что изображение по Лапласу выходного сигнала равно $Y(p)=W(p)G(p)$, значит
\begin{align*}
	Y(p)&=\frac1{p+0{,}5}*1=\frac1{p+0{,}5}\\
	y(t)&=e^{-0{,}5t}
\end{align*}
В таблице~\ref{dirak_t} указаны значения функции, график функции $\omega(t)$ изображен на рисунке~\ref{dirak}.

\begin{table}[h!]
	\caption{Таблица значений функции для синусоидального выходного сигнала}
	\label{dirak_t}
	\scalefont{.7}
	\begin{tabular}{*{14}{|c}|}
	\hline 
	t&$0$&$0{,}5$&$1$&$1{,}5$&$2$&$2{,}5$&$3$&$3{,}5$&$4$&$4{,}5$&$5$&$5{,}5$&$6$ \\
	\hline
	$y(t)$&$1$&$0{,}7788$&$0{,}6065$&$0{,}4723$&$0{,}3678$&$0{,}2865$&$0{,}2231$&$0{,}1737$&$0{,}1353$&$0{,}1053$&$0{,}0820$&$0{,}0639$&$0{,}0497$ \\
	\hline
\end{tabular}		
\end{table}

\begin{figure}[h!]
		\centering
		\includegraphics[scale=.7]{dirak.jpg}
		\caption{График весовой функции $\omega(t)$}
		\label{dirak}
\end{figure}

\subsection{Синусоидальный сигнал}
Изображение по Лапласу для синусоидального сигнала:$G(p)=\frac{\omega}{p^2+\omega^2}$.

Известно, что изображение по Лапласу выходного сигнала равно $Y(p)=W(p)G(p)$, значит
\begin{align*}
	Y(p)&=\frac1{p+0{,}5}*\frac{\omega}{p^2+\omega^2}\\
	y(t)&=\frac{e^{-0{,}5t}\omega}{\omega^2+0{,}25}-\frac{\omega\cos(\omega{}t)}{\omega^2+0.25}+\frac{\sin(\omega{}t)}{2\omega^2+0{,}5}
\end{align*}

В таблице~\ref{sin_t} указаны значения функции, график функции $\sin(\omega{}t)$ для $\omega=0{,}5$ изображен на рисунке~\ref{sin}.
\begin{table}[h!]
	\caption{Таблица значений функции для синусоидального входного сигнала}
	\label{sin_t}
	\scalefont{.7}
	\begin{tabular}{*{14}{|c}|}
	\hline 
	t&$0$&$2$&$4$&$6$&$8$&$10$&$12$&$14$&$16$&$18$&$20$&$22$&$24$ \\
	\hline
	$y(t)$&$0$&$0{,}4917$&$1{,}2902$&$1{,}6798$&$1{,}3521$&$0{,}4153$&$-0{,}7069$&$-1{,}4931$&$-1{,}5757$&$-0{,}9167$&$0{,}1735$&$1{,}1823$&$1{,}6350$ \\
	\hline
\end{tabular}		
\end{table}

\begin{figure}[h!]
		\centering
		\includegraphics[scale=.3]{sin.jpg}
		\caption{График синусоидального выходного сигнала}
		\label{sin}
\end{figure}

Частотные характеристики могут быть получены из соотношений:
\begin{align*}
	&W{j\omega} = u(\omega)+jv(\omega)= \\
	&=\frac1{j\omega+0{,}5} =\frac{0{,}5}{0{,}25+\omega^2}-j\frac{\omega}{0{,}25+\omega^2} =\\
	&=\frac2{4\omega^2+1}-j\frac{4\omega}{4\omega^2+1} \\
	&A(\omega)=\sqrt{\left(\frac{0{,}5}{0{,}25+\omega^2}\right)^2+\left(\frac{\omega}{0{,}25+\omega^2}\right)^2} \\
	&L(\omega)=20\lg2-20\lg(\sqrt{1+4\omega^2}) \\
	&\phi(\omega)=-\arctan2\omega
\end{align*}
\section{Графики функций, полученные экспериментальным путем}
\begin{figure}[h!]
		\centering
		\includegraphics[scale=.6]{dir.jpg}
		\caption{График реакции звена на $\delta$-импульс}
\end{figure}
\begin{figure}[h!]
		\centering
		\includegraphics[scale=.6]{st.jpg}
		\caption{График реакции звена на единичное воздействие}
\end{figure}
\begin{figure}[h!]
		\centering
		\includegraphics[scale=.6]{lh.jpg}
		\caption{Амплитудно-частотные характеристики звена для сигнала $\sin\omega{}t$}
\end{figure}

\pagebreak
\section{Соотношение вход/выход для сигнала $\sin\omega{}t$ на разных частотах}
Здесь входящий всегда имеет еденичную амплитуду.
\begin{figure}[h!]
		\centering
		\includegraphics[scale=.6]{lowfr.jpg}
		\caption{График сравнения при низкой частоте входного сигнала $\omega=0{,}1$}
\end{figure}
\begin{figure}[h!]
		\centering
		\includegraphics[scale=.7]{midfr.jpg}
		\caption{График сравнения при средней частоте входного сигнала $\omega=0{,}35$}
\end{figure}
\begin{figure}[h!]
		\centering
		\includegraphics[scale=.6]{hifr.jpg}
		\caption{График сравнения при высокой частоте входного сигнала $\omega=1$}
\end{figure}
\section{Вывод}
В данной лабораторной работе было смоделированно апереодическое звено 1-го порядка. Инерционными звеньями первого порядка являются конструктивные элементы, которые могут накапливать энергию или вещество, и обладающие свойством без изменения внешних воздействий приходить в установившееся состояние (самовыравниванием). При скачке воздействия выходная величина не может измениться скачком, а изменяется плавно по экспоненте, т.е. звено обладает инерцией. Отсюда происходит название звена – инерционное. Переходная функция возрастает монотонно, без колебаний. Отсюда происходит название звена – апериодическое (т.е. не имеющее периода, неколебательное). Такое звено увеличивает амплитуду низких частот синусоидального сигнала и уменьшает амплитуду высоких.